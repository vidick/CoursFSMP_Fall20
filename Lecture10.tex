\chapter{A test for $n$ qubits}

Recall from the last lecture that we reduced the proof of $\MIP^*=\RE$ to showing a form of ``$n$-qubit test'', stated as Claim~\ref{claim:pbt-2} and reformulated as a theorem here:

\begin{theorem}\label{thm:pbt}
There is a family of games $\{G_n\}_{n\geq 1}$ with the following properties. For each $n\geq 1$, in the game $G_n$ there are questions of the form $(X,a)$ and $(Z,b)$ where $X,Z$ are labels and $a,b\in\{0,1\}^n$ such that the answer expected from a player upon such a question is a single bit. In addition there are two questions $X$ and $Z$ such that the expected answer on such a question is $n$ bits long. Furthermore, for each $n\geq 1$ and $\eps>0$, for any strategy $(\ket{\psi},\{A^x_a\},\{B^y_b\})$ that succeeds with probability tat least $1-\eps$ in the game $G_n$ there is an isometry $V_A:\mH_\reg{A} \to (\C^2)^{\otimes n} \otimes \mH'_{\reg{A}}$ such that, if $X(a)$, $Z(b)$ and $\{A^X_a\}_a,\{A^Z_b\}_b$ are Alice's observables and POVM on the aforementioned questions then 
\begin{align}
 \max\Big\{&\Es{a} \big\| (V_A X(a) - (\sigma_X(a)\otimes \Id_{\reg{A}'}) V_A )\otimes \Id_\reg{B} \ket{\psi} \big\|^2\;,\notag\\
&\quad \Es{b}\big\| (V_A Z(b) - (\sigma_Z(b)\otimes \Id_{\reg{A}'}) V_A )\otimes \Id_\reg{B} \ket{\psi} \big\|^2 \Big\} \,\leq\, O(\eps^d)\;,\label{eq:pbt-obs}
\end{align}
where the expectation is taken over uniformly random $a,b\in\{0,1\}^n$, and letting $\{\sigma^X_a\}_a$ and $\{\sigma^Z_b\}$ denote POVMs representing an $n$-qubit measurement in the Hadamard and computational basis respectively,
\begin{align}
 \max\Big\{& \sum_a \big\| (V_A A^X_a  - (\sigma^X_a\otimes \Id_{\reg{A}'}) V_A )\otimes \Id_{\reg{B}}\ket{\psi} \big\|^2 \;,\notag\\
&\quad  \sum_b \big\| (V_A A^Z_b  - (\sigma^Z_b\otimes \Id_{\reg{A}'}) V_A )\otimes \Id_{\reg{B}}\ket{\psi} \big\|^2 \Big\} \,\leq\, O(\eps^d)\;,\label{eq:pbt-meas}
\end{align}
for some universal constant $d>0$. A similar statement holds for Bob's observables. Finally, there is a state $\ket{aux} \in \mH'_A \otimes \mH'_B$ such that 
\begin{equation}\label{eq:pbt-state}
 \big\| V_A \otimes V_B \ket{\psi} - \ket{\phi^+}^{\otimes n} \ket{aux} \big\|^2 \,\leq\, O(\eps^d)\;,
\end{equation}
where recall that $\ket{\phi^+} \in \C^2 \otimes \C^2$ denotes the state of an EPR pair. 
\end{theorem}

\begin{remark}\label{rk:symmetric}
The games $G_n$ in Theorem~\ref{thm:pbt}, as well as all games considered in this lecture, are \emph{symmetric}: $\mX=\mY$ and $\mA=\mB$, the distribution on questions $\pi$ is invariant under permutation of the two questions, $\pi(x,y)=\pi(y,x)$ for all $(x,y)$, and the verification predicate is symmetric as well, i.e.\ $R_n(a,b,x,y)=R_n(b,a,y,x)$ for all $(x,y,a,b)$. Define a strategy $(\ket{\psi},\{A^x_a\},\{B^y_b\})$ to be \emph{symmetric} if $\mH_\reg{A}=\mH_\reg{B}$, $\ket{\psi}$ is invariant under exchange of the two subsystems, and $A^x_a = B^x_a$ for all $x,a$. 
It is not hard to verify that whenever a game $G$ is symmetric then for any strategy $(\ket{\psi},\{A^x_a\},\{B^y_b\})$ that succeeds with some probability $1-\eps$ in the game there is a symmetric strategy $(\ket{\tilde{\psi}},\{\tilde{A}^x_a\})$ that succeeds with the same probability, such that moreover ``rigidity'' statements such as the conclusion of Theorem~\ref{thm:pbt} can be ``lifted'' from the symmetric strategy back to the original strategy. (The symmetrized strategy uses a state 
\[ \ket{\tilde{\psi}} = \ket{\psi}_{\reg{AB}} \otimes \frac{1}{\sqrt{2}}\big( \ket{0}_{\reg{A}'}\ket{1}_{\reg{B}'} +  \ket{1}_{\reg{A}'}\ket{0}_{\reg{B}'}\big) \in \mH_\reg{A} \otimes \mH_\reg{B} \otimes \C^2_{\reg{A}'} \otimes \C^2_{\reg{B}'}\;,\]
and each player uses its additional qubit to decide whether to compute its answer $a$ to question $x$ using the ``Alice'' POVM $\{A^x_a\}$ or the ``Bob'' POVM $\{B^x_a\}$; formally, for all $x,a$, $\tilde{A}^x_a = A^x_a \otimes \proj{0} + B^x_a \otimes \proj{1}$.)   
Due to this observation in the lecture we limit ourselves to the analysis of symmetric strategies. 
\end{remark}

There is one important aspect in which the theorem falls short of what we needed in the previous lecture: its verification procedure is too complex. In the theorem, $n$ qubits are being tested using ---as we will see--- questions of length $O(n)$. In order for the test to be of use in the design of the compression procedure (Claim~\ref{claim:compression}) we need the test verifier to run in time $\ll n$, so in particular questions cannot be of length $O(n)$. Achieving such a test is a challenge beyond what we can expose in a single lecture, and so we will focus on the less efficient version provided by Theorem~\ref{thm:pbt}. The more efficient version combines the kind of techniques introduced in this lecture with techniques similar to the use of polynomial error-correcting codes in the proof of the PCP theorem. For details we refer to~\cite{ji2020quantum} and~\cite[Appendix A]{ji2020mip}. 

Our strategy for the proof of Theorem~\ref{thm:pbt} is to derive it from two ingredients. Firstly, in Section~\ref{sec:approx-group} we introduce a general theory of approximate group representations and note that the consequences of Theorem~\ref{thm:pbt} follow provided that the players' strategy is, in some sense, an approximate representation of the $n$-qubit Pauli group (a finite group which we will define). Secondly, in Section~\ref{sec:nqubit-test} we design a game, or ``test'', that enforces that any successful strategy for the players in the game specifies an approximate representation of the $n$-qubit Pauli group, so that the results from the first part can be applied to it to conclude the theorem. 

\section{Approximate group representations}
\label{sec:approx-group}

\subsection{Definitions}

We first make a small detour through the theory of group representations. For $d$-dimensional matrices  $A,B$ and $\sigma$ such that $\sigma$ is positive semidefinite, write 
$$\langle A,B\rangle_\sigma = \mathrm{Tr}(AB^* \sigma)\;,$$
where we use $B^*$ to denote the conjugate-transpose. Note that the matrix trace inner product is recovered for $\sigma = \Id$. If $\sigma$ is the totally mixed state, then we obtain a dimension-normalized variant of the trace inner product. We will also write $\|A\|_\sigma = \langle A,A\rangle_\sigma^{1/2}$. 

Given an arbitrary finite group $G$ (not necessarily abelian), a group representation of $G$ is a map $f:G \to U_d(\C)$, the group of $d\times d$ unitary matrices, such that $f$ is a homomorphism: for any $x,y\in G$, $f(x^{-1}y)=f(x)^* f(y)$, where we used $^*$ to denote the conjugate transpose (which, for unitary matrices, corresponds to taking the inverse). The following definition introduces a notion of \emph{approximate} group representation.  

\begin{definition}\label{def:approx-rep}
Given a finite group $G$, an integer $d\geq 1$, $\eps\geq 0$, and a $d$-dimensional positive semidefinite matrix $\sigma$ with trace $1$, an $(\eps,\sigma)$-representation of $G$ is a function $f: G \to U_d(\C)$, the unitary group of $d\times d$ matrices, such that 
\begin{equation}\label{eq:gh-condition}
\Es{x,y\in G} \,\Re\big(\big\langle f(x)^*f(y) ,f(x^{-1}y) \big\rangle_\sigma\big) \,\geq\, 1-\eps\;,
\end{equation} 
where the expectation is taken under the uniform distribution over $G$.
\end{definition}

Note that the condition~\eqref{eq:gh-condition} is equivalent to 
\begin{equation}\label{eq:gh-condition-2}
\Es{x,y\in G} \, \big\| f(x^{-1}y) - f(x)^*f(y) \big\|_\sigma^2 \,\leq\, 2\eps\;.
\end{equation}
Taking $\eps=0$ and $\sigma$ any invertible positive definite matrix, we see that the case $\eps=0$ corresponds to an \emph{exact} representation of $G$. 

\begin{example}\label{ex:wh-1}
Consider the Weyl-Heisenberg group $\mP$, which is the group generated by the Pauli $\sigma_X$ and $\sigma_Z$ matrices. It is not hard to verify that this group has $8$ elements, which can be decomposed as $(-1)^c \sigma_X^a \sigma_Z^b$ for $a,b,c\in\{0,1\}$. 
A qubit $(X,Z)$ according to our first definition (Definition~\ref{def:qubit-take1}) can be used to specify a $(0,\sigma)$ representation of $\mP$ for \emph{any} $\sigma$ as follows:
\begin{equation}\label{eq:def-rep-pauli}
 f\big((-1)^c \sigma_X^a \sigma_Z^b\big) \,=\, (-1)^c X^a Z^b\;,
\end{equation}
for all $a,b,c\in\{0,1\}$. It is immediate to verify that for all $(x,y)\in \mP$ we have $f(x)^*f(y)=f(x^{-1}y)$ and so~\eqref{eq:gh-condition} holds with $\eps=0$ for any $\sigma$.
\end{example}

The example makes explicit the connection between our notion of qubit with group representation theory. In this lecture we will leverage the connection to make use of powerful theorems from representation theory towards the analysis of an $n$-qubit test. As a warm-up, 
the following simple and recommended exercise asks you to generalize the example to the case of a single $\eps$-approximate qubit. 

\begin{exercise}\label{ex:wh-2}
Let $(\ket{\psi},X,Z)$ be a qubit such that $\ket{\psi}\in \mH$. Define $f:\mP \mapsto U(\mH)$ as in~\eqref{eq:def-rep-pauli}. Show that $f$ is an $(0,\sigma)$-representation of $\mP$ for $\sigma = \proj{{\psi}}$. Now suppose that $(\ket{\psi},X,Z)$ is only an $\eps$-approximate qubit in the sense of Exercise~\ref{ex:approx-qubit}. Show that $f$ is an $(O(\sqrt{\eps}),\sigma)$-representation of $\mP$. 
\end{exercise}

\begin{remark}
The condition \eqref{eq:gh-condition} in Definition~\ref{def:approx-rep} is very closely related to Gowers' $U^2$ norm
$$\|f\|_{U^2}^4 \,=\, \Es{xy^{-1}=zw^{-1}}\, \big\langle f(x)f(y)^* ,f(z)f(w)^* \big\rangle_\sigma.$$
While a large Gowers norm implies closeness to an affine function, we are interested in testing homomorphisms, and the condition \eqref{eq:gh-condition} will arise naturally from our calculations in the next section. 
\end{remark}

\subsection{The Gowers-Hatami theorem}

There are many possible notions of approximate group representation. Traditionally the most frequently considered one replaces the norm in Definition~\ref{def:approx-rep} by the operator norm. An inconvenience of that variant is that in general approximate representations are not always close to exact representations (see, for example, the famous problem on ``approximately commuting'' versus ``nearly commuting'' operators). In contrast 
Gowers and Hatami~\cite{gowers2017inverse} showed that in the case of Definition~\ref{def:approx-rep}, approximate group representations can always be ``rounded'' to a nearby exact representation.\footnote{We are barely scratching the surface of a growing theory of ``stability'' for group homomorphisms; see e.g.~\cite{becker2020stability} for an introduction and discussion of related notions.} 
We state and prove a slightly more general, but quantitatively weaker, variant of the Gowers-Hatami result.

\begin{theorem}[Gowers-Hatami]\label{thm:gh}
Let $G$ be a finite group, $\eps\geq 0$, and $f:G\to U_d(\C)$ an $(\eps,\sigma)$-representation of $G$. Then there exists a $d'\geq d$, an isometry $V:\C^d\to \C^{d'}$, and a representation $g:G\to U_{d'}(\C)$ such that 
$$\Es{x\in G}\, \big\| f(x) - V^*g(x)V \big\|_\sigma^2\, \leq\, 2\,\eps.$$ 
\end{theorem}

Gowers and Hatami limit themselves to the case of $\sigma = d^{-1}I_d$, which corresponds to the dimension-normalized Frobenius norm. In this scenario they in addition obtain a tight control of the dimension $d'$, and show that one can always take $d'\ = (1+O(\eps))d$ in the theorem. We will see a much shorter proof than theirs (the proof is implicit in their argument) that does not seem to allow to recover this estimate. The extension to general $\sigma$, however, will be necessary for our purposes.

Note that  Theorem \ref{thm:gh} does not in general hold  with $d'=d$. The reason is that it is possible for $G$ to have an approximate representation in some dimension $d$, but no exact representation of the same dimension: to obtain an example of this, take any group $G$ that has all non-trivial irreducible representations of large enough dimension, and create an approximate representation in e.g. dimension one less by ``cutting off'' one row and column from an exact representation. For sufficiently ``smooth'' $\sigma$ (no disproportionately large singular values) the dimension normalization induced by the norm $\|\cdot\|_\sigma$ will make this barely  noticeable, but it will be impossible to ``round'' the approximate representation obtained to an exact one without modifying the dimension. 

\begin{example}\label{ex:wh-3}
Continuing with Example~\ref{ex:wh-1} we consider the example of $G= \mP$. In Example~\ref{ex:wh-1} we observed that a qubit $(\ket{\psi},X,Z)$ can be used to specify a $(0,\sigma)$ representation $f$ of $\mP$ such that moreover $f(-1)=-\Id$. We now check that the converse holds: for any $(0,\sigma)$-representation of $\mP$ for invertible $\sigma$, if $X=f(\sigma_X)$ and $Z=f(\sigma_Z)$ then using~\eqref{eq:gh-condition-2}, taking $x=y=\sigma_X$ and $x=y=\sigma_Z$ it follows that $X^2 = Z^2=f(1)=\Id$, and taking $x=\sigma_X$ and $y=\sigma_Z$ we get that $XZ=f(\sigma_X \sigma_Z)$ while 
\[ZX=f(\sigma_Z\sigma_X)=f(-\sigma_X\sigma_Z)=f(-1)f(\sigma_X\sigma_Z)\;,\]
where the second equality uses that the Pauli anti-commute and the last equality again uses~\eqref{eq:gh-condition-2}. Thus if $f(-1)=-\Id$ then 
 $\{X,Z\}=0$, so the $(0,\sigma)$-representation $f$ specifies a qubit and in particular Lemma~\ref{lem:qubit-2-rigid} on the structure of a qubit applies. As a result, we have shown that there exists a single  representation of $\mP$ such that $f(-1)=-\Id$, and that it is given by the Pauli matrices in dimension $2$. 
\footnote{The condition $f(-1)=-\Id$ is necessary, as there are four $1$-dimensional representations of $\mP$: all combinations $f(\sigma_X)=\pm 1$ and $f(\sigma_Z)=\pm 1$. We have found the right number of irreps: $1\cdot 2^2 + 4\cdot 1 = 8 = |\mP|$.}
\end{example}

\begin{exercise}
Using Exercise~\ref{ex:wh-2} and Example~\ref{ex:wh-3}, show that Theorem~\ref{thm:gh} for the case where $G=\mP$ implies Exercise~\ref{ex:approx-qubit}. 
\end{exercise}

\begin{exercise}
Show Theorem~\ref{thm:gh} for the case where $G=\mP$. \emph{[Hint: Adapt the proof ``by calculation''
of Proposition~\ref{prop:explicit-iso} to take into account $\eps$-approximations.]}
\end{exercise}

The main ingredient for the proof of Theorem~\ref{thm:gh} is an appropriate notion of Fourier transform over non-abelian groups. Given an irreducible representation $\rho: G \to U_{d_\rho}(\C)$, define 
\begin{equation}\label{eq:fourier}
 \hat{f}(\rho) \,=\, \Es{x\in G} \,f(x) \otimes \overline{\rho(x)}.
\end{equation}
In case $G$ is abelian, we always have $d_\rho=1$, the tensor product is a product, and \eqref{eq:fourier} reduces to the usual definition of Fourier coefficient. The only properties we will need of irreducible representations is that they satisfy the relation
\begin{equation}\label{eq:ortho}
\sum_\rho \,d_\rho\,\mathrm{Tr}(\rho(x)) \,=\, |G|\delta_{xe}\;,
\end{equation}
for any $x\in G$. Note that plugging in $x=e$ (the identity element in $G$) yields $\sum_\rho d_\rho^2= |G|$. 

\begin{proof}[Proof of Theorem \ref{thm:gh}]
Our first step is to define an isometry $V:\C^d \to \C^d \otimes (\oplus_\rho \C^{d_\rho} \otimes \C^{d_\rho})$ by
$$ V :\;u \in \C^d \,\mapsto\, \bigoplus_\rho \,d_\rho^{1/2} \sum_{i=1}^{d_\rho} \,\big(\hat{f}(\rho) (u\otimes e_i)\big) \otimes e_i,$$
where the direct sum ranges over all irreducible representations $\rho$ of $G$ and $\{e_i\}$ is the canonical basis.\footnote{Observe that this expression directly generalizes~\eqref{eq:explicit-isometry}.}
Note what $V$ does: it ``embeds'' any vector $u\in \C^d$ into a direct sum, over irreducible representations $\rho$, of a $d$-dimensional vector of $d_\rho\times d_\rho$ matrices. Each (matrix) entry of this vector can be thought of as the Fourier coefficient of the corresponding entry of the vector $f(x)u$ associated with $\rho$. 
The fact that $V$ is an isometry follows from the appropriate extension of Parseval's formula:  
\begin{eqnarray*}
& V^* V &= \sum_\rho d_\rho \sum_i (I\otimes e_i^*) \hat{f}(\rho)^*\hat{f}(\rho) (I\otimes e_i)\\
&&= \Es{x,y}\,  f(x)^*f(y) \sum_\rho d_\rho \sum_i  (e_i^* \rho(x)^T \overline{\rho(y)} e_i)\\
&&= \sum_\rho \frac{d_\rho^2}{|G|}I = I,
\end{eqnarray*}
where for the second line we used the definition \eqref{eq:fourier} of $\hat{f}(\rho)$ and  for the third we used \eqref{eq:ortho} and the fact that $f$ takes values in the unitary group.

Next define
$$g(x) = \bigoplus_\rho \,\big(I_d \otimes I_{d_\rho} \otimes \rho(x)\big), $$
a direct sum over all irreducible representations of $G$ (hence itself a representation). Lets' first compute the ``pull-back'' of $g$ by $V$: following a similar calculation as above, for any $x\in G$, 
\begin{eqnarray*}
& V^*g(x) V  &=  \sum_{\rho}  d_\rho \sum_{i,j} (I\otimes e_i^*)\hat{f}(\rho)^* \hat{f}(\rho)(I\otimes e_j) \otimes e_i^* \rho(x) e_j ) \\
&& =  \Es{z,y}\,  f(z)^*f(y)  \sum_{\rho}  d_\rho \sum_{i,j} (e_i^* \rho(z)^T \overline{\rho(y)} e_j) \big( e_i^* \rho(x) e_j \big) \\
&& =  \Es{z,y}\,  f(z)^*f(y)  \sum_{\rho}  d_\rho \mathrm{Tr}\big( \rho(z)^T \overline{\rho(y)}  {\rho(x)^T} \big) \\
&& =  \Es{z,y}\,  f(z)^*f(y)  \sum_{\rho}  d_\rho \mathrm{Tr}\big( \rho(z^{-1}y x^{-1}) \big) \\
&& =  \Es{z}\,  f(z)^*f(zx) , 
\end{eqnarray*}
where the last equality uses \eqref{eq:ortho}.
It then follows that 
\begin{eqnarray*}
&\Es{x}\, \big\langle f(x), V^*g(x) V \big\rangle_\sigma &=  \Es{x,z} \mathrm{Tr}\big( f(x) f(zx)^* f(z)\sigma\big).
\end{eqnarray*}  
This relates correlation of $f$ with $V^*gV$ to the quality of $f$ as an approximate representation and proves the theorem. 
\end{proof}

\subsection{Application: rigidity for the Magic Square game}

Recall the Magic Square game from Section~\ref{sec:ms-game}. In Lemma~\ref{lem:ms-perfect} we analyzed perfect strategies in this game and showed that any perfect strategy must ``have a qubit''. Remembering the proof, we had seen that Bob's observables in a perfect strategy must form an operator solution to the underlying system of equations, and that any such operator solution must contain two anti-commuting observables. Using Theorem~\ref{thm:gh} we can extend our earlier result to the case of approximate strategies with very little extra work. (This was first shown in~\cite{wu2016device} using a more direct proof.)

\begin{theorem}\label{thm:rigid-ms}
Let $\eps>0$, and suppose that a strategy for the players  in the Magic Square game, using  a bipartite state $\ket{\psi}\in\C^d\otimes \C^d$ and observables $B_1,\ldots,B_9$ for Bob succeeds with probability $1-\eps$, for some $\eps \geq 0$. Then $(\ket{\psi},B_2,B_4)$ is an $O(\sqrt{\eps})$-approximate qubit. Moreover, there are local isometries $V_A,V_B:\C^d \to \C^2 \otimes \C^{d'}$ such that 
\begin{equation}\label{eq:chsh-state}
\big\| V_A \otimes V_B \ket{\psi} - \ket{\phi^+} \otimes \ket{\psi'} \big\| = O(\sqrt{\eps}) \;,
\end{equation}
and 
\begin{equation}\label{eq:chsh-alice}
\big\|( V_A \otimes V_B)( \Id \otimes B_2 )\ket{\psi} - (\sigma_Z \otimes \Id) \ket{\phi^+} \otimes \ket{\psi'} \big\| \,=\, O\big(\sqrt{\eps}\big)\;,
\end{equation}
and a similar relation holds with $B_2$ replaced by $B_4$ and $\sigma_Z$ replaced by $\sigma_X$. 
\end{theorem}

Note that this is a slightly weaker variant of Theorem~\ref{thm:ms-robust} that we stated without proof, because it only characterizes a single qubit of the strategy, instead of two for the case of Theorem~\ref{thm:ms-robust}. This will suffice for our purposes. 

\begin{proof}
For the first step of the proof we follow the proof of Claim~\ref{claim:ms-ac}, except that all equalities must be made approximate equalities. 
Assume without loss of generality that Alice's strategy is specified by six $4$-outcomes projective measurements $\{A^i_{a_1,a_2,a_3}\}$, where $a_1,a_2,a_3\in\{\pm 1\}$ range over the four possible assignments that satisfy the constraint associated with the $i$-th row $(i\in \{1,2,3\})$ or $(i-3)$-th column ($i\in \{4,5,6\}$). (We can assume that Alice always returns a valid assignment because she knows that she will lose if not, so we do not even need to include a symbol for such evidently wrong answers in the game.)

For any $i\in\{1,\ldots,9\}$, in addition to Bob's observable $B_i$ associated with question $i$ we define two observables from Alice's strategy, obtained by recording her answer associated to entry $i$ when she is asked the unique row containing $i$ --- call this observable $C_i$ --- or the unique column containing $i$ --- call this observable $C'_i$. Formally, $C_1 = \sum_{a_1,a_2,a_3\in\{\pm 1\}}a_1 A^1_{a_1,a_2,a_3}$, and similar relations can be used to define each $C_i$ and $C'_i$. Due to the parity constraints enforced on Alice's answers it is always the case that $C_1C_2C_3=+\Id,\ldots,C_3C_6C_9=-\Id$.
 
By definition, success of the strategy in the game implies $18$ equations of the form 
\begin{equation}\label{eq:ms-bc-1}
 \bra{\psi} C_i \otimes B_i \ket{\psi} \,\geq \, 1-18\eps\qquad\text{and}\qquad \bra{\psi} C'_i \otimes B_i \ket{\psi} \,\geq \, 1-18\eps\;,
\end{equation}
for all $i\in \{1,\ldots,9\}$. Indeed, each such equation represents the probability that the players return valid answers, conditioned on each of the $18$ possible pairs of questions in the game. These relations allow us to mimic the proof of Claim~\ref{claim:ms-ac}, as follows: 
\begin{align*}
B_2 B_4 \ket{\psi} & \approx C_4 \otimes B_2 \ket{\psi}\\
&= C_5 C_6 \otimes B_2 \ket{\psi}\\
&\approx C_5 C_6 C_2 \otimes \Id \ket{\psi}\\
&= C_5 C_6 C_3 C_1 \otimes \Id \ket{\psi}\\
&\approx C_5C_6C_3 \otimes B_1 \ket{\psi}\\
&\approx C_5C_6\otimes B_1 B_3 \ket{\psi}\\
&\approx C_5C'_6\otimes B_1 B_3 \ket{\psi}\\
&\approx C_5C'_6 C'_3\otimes B_1  \ket{\psi}\\
&=  C_5C'_9\otimes B_1  \ket{\psi}\;,
\end{align*}
where here we use the notation $\ket{u}\approx \ket{v}$ to mean $\|\ket{u} - \ket{v}\|^2 = O(\eps)$. Here, each of the approximations is obtained by bounding the squared norm of the difference using the Cauchy-Scharz inequality and the required relation; for example, for the first approximation we write
\begin{align*}
 \big\|(\Id \otimes B_2 B_4 - C_4 \otimes B_2)\ket{\psi}\big\|^2
&= \big\|(\Id \otimes B_2)(\Id\otimes B_4 - C_4 \otimes \Id)\ket{\psi}\big\|^2\\
&= \big\|(\Id\otimes B_4 - C_4 \otimes \Id)\ket{\psi}\big\|^2\\ 
&= 2 - 2 \bra{\psi} B_4 \otimes C_4 \ket{\psi} \\
& \leq 36\eps\;,
\end{align*}
where the derivation uses that $B_2,B_4$ and $C_4$ are Hermitian and square to identity. 
Using a similar chain of approximations starting from $B_4 B_2 \ket{\psi}$, it follows that $(\ket{\psi},B_2,B_4)$ is an $O(\sqrt{\eps})$-approximate qubit.
 
As already observed in Exercise~\ref{ex:wh-2}, it follows that $B_2$ and $B_4$ induce an approximate representation of $\mP$ by setting 
$$ f(\pm \Id)=\pm\Id,\quad f(\pm\sigma_Z)=\pm B_2,\quad\quad f(\pm\sigma_X)=\pm B_4,\quad f(\pm\sigma_X\sigma_Z) = \pm B_4B_2\;.$$
Note that this is a legal definition, since $B_2$, $B_4$, and $B_2B_4$ are all unitary. Moreover, using only the approximate anti-commutation and the fact that $B_2$ and $B_4$ are observables it is immediate to verify that the conditions of Theorem~\ref{thm:gh} are satisfied, i.e. $f$ is an $(O({\eps}),\rho_\reg{B})$-representation of $\mP$, where $\rho_\reg{B}$ denotes the reduced density of $\ket{\psi}$ on $\mH_\reg{B}$.  

Applying the theorem, there must exist an exact representation $g$ of $\mP$ to which $f$ is close. However, as we saw in Example~\ref{ex:wh-3} the representation theory of $\mP$ is not complicated. It has four $1$-dimensional irreducible representations, but all of them map $-\Id$ to $1$, so they cannot be close to $f$. Since any representation is a direct sum of irreducible representations, and all irreducible representations of $\mP$ are far from $f$, $g$ must be a direct sum of multiple copies of the unique irreducible $2$-dimensional representation of $\mP$, which is precisely given by the Pauli matrices, together with possibly a small (relative to the total dimension) number of $1$-dimensional relations. Ignoring the presence of such representations for simplicity,\footnote{To account for them we would select the ``corner'' of the range of the isometry $V$ that includes only copies of the $2$-dimensional irrep; this is a simple technicality.} $g(\sigma_Z) = \sigma_Z \otimes \Id$ and $g(\sigma_X) = \sigma_X \otimes \Id$, which gives~\eqref{eq:chsh-alice}. 

To conclude~\eqref{eq:chsh-state} we observe that the relations~\eqref{eq:ms-bc-1} imply that also $(\rho_\reg{A},C_2,C_4)$ is an $O(\sqrt{\eps})$-approximate qubit. This allows us to define an approximate representation of $\mP$ on $\mH_\reg{A}$, and apply Theorem~\ref{thm:gh} again to obtain on isometry $V_\reg{A}$ which maps $(\rho_\reg{A},C_2,C_4)$ close to an exact qubit. Letting 
\[ \ket{\psi'} = V_\reg{A} \otimes V_\reg{B} \ket{\psi} \in (\C^2 \otimes \mH_\reg{A}')\otimes (\C^2 \otimes \mH_\reg{B}') \]
we see from~\eqref{eq:ms-bc-1} and~\eqref{eq:chsh-alice} that 
\[\big| \bra{\psi} \big(\sigma_X \otimes \sigma_X \otimes \Id_{\reg{A}'\reg{B}'} + \sigma_Z \otimes \sigma_Z \otimes \Id_{\reg{A}'\reg{B}'} \big) \ket{\psi}\big| \geq 1-O({\eps})\;.\]
To conclude note that $\frac{1}{2}(\sigma_X \otimes \sigma_X + \sigma_Z \otimes \sigma_Z )$ is an observable with a single eigenvalue $1$, with associated eigenvector $\ket{\phi^+}$, and all other eigenvalues equal to $0$ or $-1$. 
\end{proof}



\section{Testing $n$ qubits}
\label{sec:nqubit-test}

The intuition we gained from analyzing the Magic Square game provides us with a clear roadmap for the design of a test for $n$ qubits: (i) design a game such that success in the game requires the players to share observables that satisfy all relations that we expect from elements of the group generated by $n$ quits $(X_1,Z_1),\ldots,(X_n,Z_n)$, (ii) apply Theorem~\ref{thm:gh} to obtain closeness of the strategy to an exact representation of this group, and (iii) use that, hopefully, all non-trivial representations gives us what we want, i.e. $n$ exact qubits. 

We start in Section~\ref{sec:wh} by studying the underlying group of $n$ qubits. Then in Section~\ref{sec:wh-test} we design ``tests'' for the group product relation. 

\subsection{The Weyl-Heisenberg group}
\label{sec:wh}

Denote by $\mP_n$ the ``$n$-qubit Weyl-Heisenberg group,'' i.e.\ the matrix group generated by $n$-fold tensor products of single-qubit $\sigma_X$ and $\sigma_Z$ matrices. The group $\mP_n$ has cardinality $2\cdot 4^n$, and elements of $\mP_n$ have a unique representative of the form $\pm \sigma_X(a)\sigma_Z(b)$ for $a,b\in\{0,1\}^n$. 

 The irreducible representations of  $\mP_n$ are easily computed from those of $\mP$; for us the only thing that matters is that the only irreducible representation which satisfies $g(-\Id)=-g(\Id)$ has dimension $2^n$ and is given by the defining matrix representation (in fact, it is the only irreducible representation in dimension larger than $1$).

With the upcoming application to a qubit test in mind, 
we state a version of the Gowers-Hatami theorem tailored to the group $\mP_n$ and a specific choice of presentation for the group relations. 

\begin{corollary}\label{cor:gh}
Let $n,d$ be integer, $\eps \geq 0$, $\ket{\psi} \in \C^d \otimes \C^d$ a permutation-invariant state, $\sigma$ the reduced density of $\ket{\psi}$ on either system,  and $f: \{X,Z\}\times \{0,1\}^n \to U(\C^d)$. For $a,b\in\{0,1\}^n$ let $X(a)=f(X,a)$, $Z(b)=f(Z,b)$, and assume $X(a)^2=Z(b)^2=I_d$ for all $a,b$. Suppose that the following inequalities hold: consistency
\begin{equation}\label{eq:gh-cons}
 \Es{a} \, \bra{\psi} \big(X(a) \otimes X(a)\big) \ket{\psi} \,\geq\,1-\eps\;,\qquad \Es{b} \,\bra{\psi} \big(Z(b) \otimes Z(b) \big)\ket{\psi}\,\geq\,1-\eps\;,
\end{equation}
linearity
\begin{equation}\label{eq:gh-commute}
 \Es{a,a'} \,\big\|X(a)X(a')-X(a+a')\big\|_\sigma^2 \leq \eps\;,\qquad\Es{b,b'}\, \big\|Z(b)Z(b')-Z(b+b')\big\|_\sigma^2 \leq \eps\;,
\end{equation}
and anti-commutation
\begin{equation}\label{eq:gh-ac}
 \Es{a,b} \,\big\| X(a)Z(b)-(-1)^{a\cdot b} X(a)Z(b)\big\|_\sigma^2\,\leq\,\eps\;.
\end{equation}
Then there exists a $d'\geq d$, an isometry $V:\C^d\to \C^{d'}$, and a representation $g:\mP_n\to U_{d'}(\C)$ such that $g(-I)=-I_{d'}$ and
$$\Es{a,b}\, \big\| X(a)Z(b) - V^*g(\sigma_X(a)\sigma_Z(b))V \big\|_\sigma^2 \,=\, O(\eps)\;.$$
\end{corollary}

Note that the conditions \eqref{eq:gh-commute} and \eqref{eq:gh-ac} in the corollary are very similar to the conditions required of an approximate representation of the group $\mP_n$; in fact it is easy to convince oneself that their exact analogue suffices to imply all the group relations. The reason for choosing those specific relations is that they can be checked using games; see the next subsection for this. Condition \eqref{eq:gh-cons} is necessary to derive the conditions for the application of the Gowers-Hatami theorem from \eqref{eq:gh-commute} and \eqref{eq:gh-ac}, and is also testable; see the proof. The conclusion of the corollary implies the conclusion of Theorem~\ref{thm:pbt} simply by using the aforementioned fact on non-trivial representations of $\mP_n$ (see details at the end of Section~\ref{sec:th101}). 

\begin{remark}
Corollary~\ref{cor:gh} can be seen as an extension of the Blum-Luby-Rubinfeld linearity test~\cite{blum1993self}. The latter makes a similar statement, but for the commutative group $\{\pm\sigma_X(a)|\, a\in\{0,1\}^n\}$. 
\end{remark}


\begin{proof} 
To apply the Gowers-Hatami theorem we need to construct an $(\eps,\sigma)$-representation $f$ of the group $\mP_n$. Using that any element of $\mP_n$ has a unique representative of the form $\pm \sigma_X(a)\sigma_Z(b)$ for $a,b\in\{0,1\}^n$, we define $f(\pm \sigma_X(a)\sigma_Z(b)) = \pm X(a)Z(b)$. Next we need to verify that $f$ is an approximate representation. Let $x,y\in \mP_n$ be such that $x=\sigma_X(a_x)\sigma_Z(b_x)$ and $y=\sigma_X(a_y)\sigma_Z(b_y)$ for $n$-bit strings $(a_x,b_x)$ and $(a_y,b_y)$ respectively. Up to phase, we can exploit successive cancellations to decompose $(f(x)f(y)^*-f(xy^{-1}))\otimes I$ as
\begin{eqnarray*}
&&\big( X(a_x)Z(b_x)X(a_y)Z(b_y) -(-1)^{a_y\cdot b_x} X(a_x+a_y) Z(b_x+b_y)\big)\otimes I \\ 
&&\qquad =  X(a_x)Z(b_x)X(a_y)\big (Z(b_y)\otimes I - I\otimes Z(b_y)\big)\\
&& \qquad\qquad+ X(a_x)\big(Z(b_x)X(a_y) - (-1)^{a_y\cdot b_x} X(a_y)Z(b_x)\big)\otimes Z(b_y)\\
&& \qquad\qquad+(-1)^{a_y\cdot b_x} \big( X(a_x)X(a_y)\otimes Z(b_y)\big) \big( Z(b_x)\otimes I - I\otimes Z(b_x)\big)\\
&& \qquad\qquad+  (-1)^{a_y\cdot b_x}  \big( X(a_x)X(a_y)\otimes Z(b_y)Z(b_x) - X(a_x+a_y)\otimes Z(b_x+b_y)\big)\\
&& \qquad\qquad+  (-1)^{a_y\cdot b_x} \big(  X(a_x+a_y)\otimes I \big)\big(I\otimes Z(b_x+b_y) - Z(b_x+b_y)\otimes I\big).
\end{eqnarray*}
(It is worth staring at this sequence of equations for a little bit. In particular, note the ``player-switching'' that takes place in the 2nd, 4th and 6th lines; this is used as a means to ``commute'' the appropriate unitaries, and is the reason for including \eqref{eq:gh-cons} among the assumptions of the corollary.)
Evaluating each term on the state $\ket{\psi}$, taking the squared Euclidean norm, and then the expectation over uniformly random $a_x,a_y,b_x,b_y$, the inequality $\| AB\ket{\psi}\| \leq \|A\|\|B\ket{\psi}\|$ and the assumptions of the theorem let us bound the overlap of each term in the resulting summation by $O({\eps})$. Using $\| (A\otimes I) \ket{\psi}\| = \|A\|_\sigma$ by definition and the triangle inequality we have obtained the bound
$$ \Es{x,y}\,\big\|f(x)f(y)^* - f(xy^{-1})\big\|_\sigma^2 \,=\, O({\eps}).$$
We are now in a position to apply the Gowers-Hatami theorem, which gives an isometry $V$ and exact representation $g$ such that
\begin{equation}\label{eq:gi}
\Es{a,b}\,\Big\|  X(a)Z(b)- \frac{1}{2}V^*\big( g(\sigma_X(a)\sigma_Z(b))  -  g(-\sigma_X(a)\sigma_Z(b))\big)V\Big\|_\sigma^2 \,=\, O({\eps}). 
\end{equation}
Using that $g$ is a representation, $g(-\sigma_X(a)\sigma_Z(b)) = g(-I)g(\sigma_X(a)\sigma_Z(b))$. It follows from \eqref{eq:gi} that $\|g(-I) + I \|_\sigma^2 = O({\eps})$, so we may restrict the range of $V$ to the subspace where $g(-I)=-I$ without introducing much additional error. 
\end{proof}



\subsection{Testing the Weyl-Heisenberg group relations}
\label{sec:wh-test}



Corollary \ref{cor:gh} makes three assumptions about the observables $X(a)$ and $Z(b)$: that they satisfy approximate consistency \eqref{eq:gh-cons}, linearity \eqref{eq:gh-commute}, and anti-commutation \eqref{eq:gh-ac}. To complete our test, we need to show how these relations can be ``certified'' in a two-player game. There are multiple ways this can be done; we give one. We start by introducing two stand-alone ``tests'' that we later combine to define the game $G_n$. 

\begin{quote}
\textbf{Linearity test:}
\begin{itemize}
\item[(a)] The referee selects $W\in\{X,Z\}$ and $a,a'\in\{0,1\}^n$ uniformly at random. She sends $(W,a,a')$ to one player and $(W,a)$, $(W,a')$, or $(W,a+a')$ to the other.\footnote{Elements such as $(W,a)$ are labels sent to the players as their question, and carry no other intrinsic meaning.} 
\item[(b)] The first player replies with two bits, and the second with a single bit. The referee accepts if and only if the player's answers are consistent. 
\end{itemize}
\end{quote}

As always in this section, the test treats both players symmetrically. As a result (see Remark~\ref{rk:symmetric} we can assume that the players' strategy is symmetric, and is specified by a permutation-invariant state $\ket{\psi}\in \C^d \otimes \C^d$ and a measurement for each question: an observable $W(a)$ associated to questions of the form $(W,a)$, and a four-outcome measurement $\{W_{a,a'}\}$ associated with questions of the form $(W,a,a')$. 

The linearity test described above is almost identical to the BLR linearity test, except for the use of the basis label $W\in\{X,Z\}$. The following lemma states conditions that a strategy must satisfy in order to succeed with high probability in the test. 
 
\begin{lemma}\label{lem:com}
Suppose that a family of observables $\{W(a)\}$ for $W\in\{X,Z\}$ and $a\in\{0,1\}^n$, generates outcomes that succeed in the linearity test with probability $1-\eps$, when applied on a symmetric bipartite state $\ket{\psi}\in\C^d\otimes \C^d$ with reduced density matix $\sigma$. Then the following hold: approximate consistency
$$ \Es{a} \, \bra{\psi}\big(X(a) \otimes X(a)\big) \ket{\psi} \,=\,1-O(\eps),\qquad \Es{b} \, \bra{\psi}\big(Z(b) \otimes Z(b) \big)\ket{\psi}\,\geq\,1-O(\eps),$$ 
and linearity 
$$
 \Es{a,a'} \,\big\|X(a)X(a')-X(a+a')\big\|_\sigma^2 = O(\eps),\qquad\Es{b,b'}\, \big\|Z(b)Z(b')-Z(b+b')\big\|_\sigma^2  \,=\, O({\eps}).$$
\end{lemma}

\begin{exercise}
Prove the lemma. (In the case of classical strategies, the conditions are an immediate reformulation of the test. The proof for quantum strategies is not much harder.)
\end{exercise}

Testing anti-commutation can be done using the Magic Square game. 

\begin{quote}
\textbf{Anti-commutation test:}
\begin{itemize}
\item[(a)] The referee selects $a,b\in\{0,1\}^n$ uniformly at random under the condition that $a\cdot b=1$. She plays the Magic Square game with both players, with the following modifications: if the question to the second player is $2$ or $4$ she sends $(X,a)$ or $(Z,b)$ instead; in all other cases he sends the original label of the question in the Magic Square game together with both strings $a$ and $b$. 
\item[(b)] Each player provides answers as in the Magic Square game. The referee accepts if and only if the player's answers would have been accepted in the game. 
\end{itemize}
\end{quote}

Using Theorem~\ref{thm:rigid-ms} it is straightforward to show the following. 

\begin{lemma}\label{lem:ac}
Suppose a strategy for the players succeeds in the anti-commutation test with probability at least $1-\eps$, when performed on a symmetric bipartite state $\ket{\psi} \in \C^d \otimes \C^d$ with reduced density matrix $\sigma$. Then the observables $X(a)$ and $Z(b)$ applied by the player upon receipt of questions $(X,a)$ and $(Z,b)$ respectively satisfy 
\begin{equation}\label{eq:ac}
 \Es{a,b:\,a\cdot b=1} \,\big\| X(a)Z(b)-(-1)^{a\cdot b} Z(b)X(a)\big\|_\sigma^2\,=\,O\big(\sqrt{\eps}\big).
\end{equation}
\end{lemma}

\subsection{Application: an $n$-qubit test}
\label{sec:th101}

We are ready to put all the pieces together and make explicit the game $G_n$ that underlies Theorem~\ref{thm:pbt}. For historical reasons we call this game the ``$n$-qubit Pauli braiding test''. 

\begin{quote}
\textbf{$n$-qubit Pauli braiding test:}
With probability $1/3$ each, 
\begin{itemize}
\item[(a)] Execute the linearity test;
\item[(b)] Execute the anti-commutation test;
\item[(c)] Execute the following consistency test: Send one player a label $W\in\{X,Z\}$ uniformly at random, and to the other $(W,a)$ for $a\in\{0,1\}^n$ chosen uniformly at random. Expect answers $c \in \{0,1\}^n$ and $c'\in\{0,1\}$ respectively. Accept if and only if $a\cdot c = c'$. 
%\item[(c)] Execute a consistency test: select $W\in\{X,Z\}$ and $a\in\{0,1\}^n$ uniformly at random. Send $(W,a)$ to both players and accept if and only if their answers match. 
\end{itemize}
\end{quote}

\begin{proof}[Proof sketch of Theorem~\ref{thm:pbt}]
For any integer $n\geq 1$ let $G_n$ be the $n$-qubit Pauli braiding test. Suppose that a family of observables $W(a)$, for $W\in\{X,Z\}$ and $a\in\{0,1\}^n$, together with projective measurements $\{A^X_a\}_{a\in\{0,1\}^n}$ and $\{A^Z_b\}_{b\in\{0,1\}^n}$ and a state $\ket{\psi}\in\C^d\otimes \C^d$ specify a symmetric strategy that succeeds in with probability at least $1-\eps$ in the game. 

Using Lemma \ref{lem:com} and Lemma \ref{lem:ac}, success with probability $1-\eps$ implies that conditions \eqref{eq:gh-cons}, \eqref{eq:gh-commute} and \eqref{eq:gh-ac} in Corollary \ref{cor:gh} are all satisfied, up to error $O(\sqrt{\eps})$. (In fact, Lemma \ref{lem:ac} only implies \eqref{eq:gh-ac} for strings $a,b$ such that $a\cdot b=1$. The condition for string such that $a\cdot b=0$ follows from the other conditions.) The conclusion of the corollary is that there exists an isometry $V$ such that the observables $X(a)$ and $Z(b)$ satisfy 
$$\Es{a,b}\, \big\| X(a)Z(b) - V^*g(\sigma_X(a)\sigma_Z(b))V \big\|_\sigma^2 \,=\, O(\sqrt{\eps})\;,$$
for some non-trivial representation $g$ of $\mP_n$. Using what we know of non-trivial representations,~\eqref{eq:pbt-obs} follows. To obtain~\eqref{eq:pbt-meas} we use success in part (c) of the test, which using $\sigma^X_a = \Es{b} (-1)^{a\cdot b} \sigma_X(b)$ and similarly for $\sigma^Z_b$ immediately translates into the desired relation. 

Finally, using the consistency relations \eqref{eq:gh-cons} that follow from part (a) of the  test together with the above we get
$$\Es{a,b}\, \bra{\psi} (V\otimes V)^* \big(  \sigma_X(a)\sigma_Z(b)\otimes  \sigma_X(a)\sigma_Z(b)\big)(V\otimes V)\ket{\psi} \,=\, 1-O(\sqrt{\eps}).$$
Using similar reasoning as the end of the proof of Theorem~\ref{thm:rigid-ms},~\eqref{eq:pbt-state} follows.
\end{proof}


